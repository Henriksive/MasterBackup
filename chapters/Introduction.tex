% !TEX root = ../Article.tex
\chapter{Introduction}
\section{Motivation}
In todays society the standard commercial smartphone is considered secure by the average user. This statement is partially true as the regular user rarely encounters sensitive data apart from personal passwords. As the development on smartphones have skyrocketed new parties have expressed interest in smartphone capabilities. In relation to this there is now a need for the smartphone to handle sensitive data, e.g, health data, military intelligence and governmental information.

Handling senstive data in context of smartphones involves storing the data, processing the data and potentially sharing the data. In these areas there exists obstacles that needs to be dealt with in order to handle the sensitive data securely. Two of the biggest obstacles are being able to encrypt and decrypt data effectively and key management/storage.

\section{Problem statement}
% Abstract?


\section{Goals}
The overall goal of this thesis is to explore and evaluate the possibilities of how smart cards and mobile devices can co-operate to achieve a higher level of information security. To better understand what limitations there are, we will need to create a framework for easy communication between a mobile device and smart card. This includes developing test applications for both the mobile device and the smart card. A framework, mobile application and smart card application allows us to test performance and security, with as close to real life use cases as possible.

To achieve the overall goal we will need to look into a set of sub-goals. The sub-goals are as follows:
\begin{enumerate}
    \item Determine limitations on the smart card.
    \item Identify use cases we are able to solve using smart cards.
    \item Outline standards for solving use cases using smart cards.
    \item Test the use cases with real-life data.
\end{enumerate}

\section{Chapter introduction}
%introduce following chapters!
