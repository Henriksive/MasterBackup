% !TEX root = ../Article.tex
\chapter{Introduction}
\section{Problem statement and motivation}
%In today's society the standard commercial smartphone is considered secure by the average user. This statement is partially true as the regular user rarely encounters sensitive data apart from personal passwords. As the development on smartphones have skyrocketed new parties have expressed interest in smartphone capabilities. In relation to this there is now a need for the smartphone to handle sensitive data, e.g.: health data, military intelligence and governmental information.
The original use of mobile phones was to make calls and send text messages. As mobile phones evolved into what we call smartphones, their possible fields of application have increased, and with that the need for better security. In today's society we use smartphones to access our bank accounts, control our cars and houses, identify ourselves, pay in shops, buy tickets and much more. Modern smart phones do implement a lot of security, but its security is mainly intended for the average private user. The innovation and potential of smartphone technology has attracted the attention of other types of users like military forces, governmental employees, health personnel, and others. These new users handle more sensitive data and perform more critical tasks than most normal users. In order to be able to adopt smartphones in these settings, their integrated security mechanisms may not always be enough, and additional technology and novel solutions are needed.

%Handling sensitive data in context of smartphones involves storing the data, processing the data and potentially sharing the data. In these areas there exist obstacles that needs to be dealt with in order to handle the sensitive data securely. Two of the biggest obstacles are being able to encrypt and decrypt data effectively and key management/storage.

%In a perfect world the smartphone would be able to overcome these obstacles by itself. Many mobile devices lack the capability to store keys securely and run operations in a secure environment and thus cannot execute all steps needed. A smart card cans provide a solution to these problems, but we have to investigate if the smart cards introduces new attack vectors and constraints.

A specific problem of smartphones, and mobile devices in general, is that they are much easier to lose or get stolen than stationary equipment. Mobile devices are relatively easy to get physical access to as they are often left unattended, even only if for a few minutes. In addition, one of the things that makes mobile devices so great, is their ability to constantly be connected to a network through either Wi-Fi, radio, Bluetooth, NFC or USB. This is a double edged sword as it offers multiple entry points that an attacker can use to get in even without physical access. One assumption one must make, is that a motivated attacker will most likely be able to compromise the device at some point. In a compromised device we cannot trust anything, not even the operating system as an attacker may have root access. This possibility is not something we can accept as sensitive data is being processed, stored and exchanged with the device. A common solution is to encrypt the data on the device, but if the device is compromised one must assume that the cryptographic keys are also compromised.

%In that case, we cannot trust anything in the device any more, including the operating system. This is not acceptable when very sensitive data being processed, stored and exchanged on these devices. Encrypting the mobile device may not be enough to protect the data, as the encryption keys are often stored on the same device for convenience.
The reason why we chose to investigate smart cards as an additional secure element is that we believe that they can increase the security of smartphones. These cards have an internal secure execution environment which can be used to generate, store and use cryptographic keys, and since they are tamper resistant, they provide protection against physical attacks. In addition, they have their own operating system that can run small customized applets which can be trusted to execute critical task securely even if the mobile device is compromised.

In particular we look at what is possible to achieve by using ``off-the-shelf'' mobile devices and smart cards. In other words, we are not interested in solutions that require a modification of the mobile device, like having to flash a customized OS in order to enable additional modules or security mechanisms.


\section{Goals}
\label{sec:goals}
The overall goal of this thesis is to explore and evaluate the possibilities of how smart cards and mobile devices can co-operate to achieve a higher level of information security. To better understand what limitations there are, we will need to create a framework for easy communication between a mobile device and smart card. This includes developing test applications for both the mobile device and the smart card. A framework, mobile application and smart card application allows us to test performance and security, with as close to real life use cases as possible.

In order to achieve this goal we will attempt to answer the following research questions:
\begin{itemize}
  \item ``What are the limitations of smart cards in the context of hardware?''
  \item ``What are the limitations of smart cards in the context of software?''
  \item ``What are the types of use cases we are able to solve and strengthen the security of using smart cards?''
\end{itemize}

%To achieve the overall goal we will need to look into a set of sub-goals. The sub-goals are as follows:
%\begin{enumerate}
%    \item Determine limitations on the smart card.
%    \item Identify use cases we are able to solve using smart cards.
%    \item Outline standards for solving use cases using smart cards.
%    \item Test the use cases with real-life data.
%\end{enumerate}


\section{Related work}
The paper ``Plug-n-Trust: Practical Trusted Sensing for mHealth'' \cite{plugntrust} discusses the possibility to use smart cards to ensure confidentiality of data from medical sensors on a patient. There are a lot of similarities in their problems statement and findings such as the need for cryptography, attack vectors and establishing trust in a limited environment. Their main area of application is to prepare and send data to a backend service (off card/device) securely. The result is that the paper's goal differs from ours when it comes down to data flow and system architecture which brings new problems to the table that we need to solve.

The paper ``Practical Attack Scenarios on Secure Element-enabled Mobile Devices'' \cite{practicalAttacksSE} presents which options exists for smart card communication and what types of attack vectors that are relevant to systems utilizing secure elements. The presented attack scenarios are highly relevant for our study and serve as a starting point for potential attack vectors on our smart cards. 
%The Secure Element Evaluation Kit (SEEK) for the Android platform is an open source project, maintained by Giesecke \& Devrient GmbH \cite{Giesecke}, which have a vision to make it easier to use secure elements in Android applications. The framework provides access to a variety of secure elements such as SIM cards, micro SD cards and embedded secure elements. Their final goal is to have their library as an integrated part of the Android operating system such that all new mobile devices comes with hardware-backed security support \cite{SEEK}. However, their focus are on the Android platform, while this thesis will focus more on the smart card itself and what a smart card brings to the table security wise.

\section{Chapter organization}
\paragraph{Chapter 1 - Introduction}
Presents a problem statement, motivation and research question for this master thesis.

\paragraph{Chapter 2 - Background}
Introduces all concepts needed for understanding how smart cards and relevant technology function. Gives a basic understanding for which threats exists for the platform.

\paragraph{Chapter 3 - Smart card framework}
Discusses the basis for the smart card communication framework, what goals we have for the framework and describes how we are going to work when developing the framework.

\paragraph{Chapter 4 - Challenges}
Identifies and examines challenges that have emerged when trying to use smart cards in co-operation with mobile devices. Presents and evaluates a solution for the problems.

\paragraph{Chapter 5 - Framework Implementation}
Gives an overview of how the framework is implemented and structured using code examples and diagrams.

\paragraph{Chapter 6 - Test cases}
Shows the testing environment for the smart cards and provides detailed testing results for smart card limitations and use cases.

\paragraph{Chapter 7 - Discussion and Conclusion}
Discusses the research questions that we have established and what experiences we encountered during this master thesis.
