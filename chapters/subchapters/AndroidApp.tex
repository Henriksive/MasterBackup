% !TEX root = ../../Article.tex
\section{Android framework}
\label{sec:androidApp}
We used the same approach on the Android framework as on the smart card application; an autonomous and easy to extend platform for tests. This resulted in a new library, ``smartcardlibrary'',  which sole purpose is to transmit APDUs as easily as possible along with some essential functionality.

\subsection{Achieving framework goals}
In section \ref{sec:designAndroidGoals} we described the goals of the framework. We wanted a framework that was easy to use, preferably requiring little to no understanding of smart cards, and at the same time be extendable. To achieve this, we abstract as much as possible of the smart card aspect, and create controllers that developers can utilize.

Figure \ref{fig:package} is a package diagram of the Android framework and the easiest way of using the framework is to only focus on the \texttt{Controller} package. More specifically one will only need to understand \texttt{CommunicationController} to be able to communicate with smart cards (either via pre-implemented methods or custom APDUs, this is described in \ref{sec:func}). To accommodate the need for more advanced functionality, or rather have the framework be extendable, developers have, if they choose, direct access to the NFC and mSD controllers in the same package.

\begin{figure}[h!]
  \captionsetup{justification=centering,margin=1.5cm}
  \caption{Library package diagram.}
  \label{fig:package}
  \centering
    \includegraphics[width=0.85\textwidth]{images/package2.png}
\end{figure}
The package diagram, figure \ref{fig:package}, shows how  the packages in the library are structured.
\newpage
\subsection{Responsibility areas}
In light of chapter \ref{ch:challenges} it is important to understand how the smart card library fit into the bigger picture and what it is responsible of. In figure \ref{fig:responsibility} we can see that there are 3 main ``layers''; the Android operating system, the Android application and the smart card library. Even though we define these layers as separate entities, we are technically incorrect when stating so, e.g., the smart card library is technically a part of the Android application and thus they are the same layer. From a more abstract perspective the separation of layers is more correct and can help give a better understanding of the architecture.

The first layer is the Android operating system. The Android operating system is in charge of persistent storage (if needed), key storage and key generation. We discussed key storage and key generation in section \ref{sec:mobileDeviceKeys} and how it should be handled. It is also worth noting that the Android operating system is responsible for running the Android application.

The second layer is the Android application. This layer is responsible for serving the user interface to the user. This may include PIN input, displaying sensitive data, etc. If the Android application handles sensitive data it is vital that it disposes the information correctly after use.

The last layer is the smart card library. The smart card library handles the incoming and outgoing communication with the smart cards. The smart card library utilizes a temporary cache to achieve asynchronous communication with smart cards and as with the Android application, it is necessary to clean out the temporary cache if sensitive data has been passing through the library.

\begin{figure}[h!]
  \captionsetup{justification=centering,margin=1.5cm}
  \caption{Diagram showing which responsibilities the layers in an Android application have.}
  \label{fig:responsibility}
  \centering
    \includegraphics[width=0.95\textwidth]{images/Component_diagram.png}
\end{figure}
\clearpage

\subsection{3rd party libraries}
\label{sec:gemaltoApp}
Gemalto provides a java library, IDGo800, for communicating and utilizing built-in methods with their smart cards.

\begin{aquote}{Gemalto.com \cite{GemaltoIDGo800}}
``IDGo 800 for Mobiles is a cryptographic middleware that supports the Gemalto IDPrime cards and Secure Elements on Mobile platforms: Contact and contactless smart cards, MicroSD cards, UICC-SIM cards, embedded Secure Elements (eSE) and Trusted Execution Environment (TEE).''
\end{aquote}

The part of IDGo800 SDK we are interested in is very small and enables us to send custom APDUs to micro SD smart cards.

We will be using the ``android.nfc'' package in order to communicate with NFC smart cards. This package is included in the standard Android SDK which in turns means that all Android devices with a NFC reader and minimum API level 9 \cite{androidNFCminSDK} can use our library.

\subsection{Framework functionality}
\label{sec:func}
The first functionality we will describe of the implementation is ``extendable'' or in other words, being able to send custom APDUs. Explaining how this is implemented will give a better understanding of how the framework is built up and makes it easier to understand the pre-implemented methods.

\paragraph{Custom APDUs}\mbox{}\\
To send custom APDUs to a smart card,\texttt{CommunicationController} must be instantiated and the application must know the application identifier of the smart card application. Further the current activity must implement\\ \texttt{NfCSmartcardControllerInterface} or \texttt{MSDSmartcardControllerInterface} (depending on smart card type) in order to be notified when the transaction is complete. Before continuing one will need to call the methods \texttt{setupNFCController} or \texttt{setupmSDController} depending on the smart card. Listing \ref{lst:NFCLibraryExample} shows an example implementation on how an activity may utilize the library for sending custom commands to a NFC smart card.
\newpage
\begin{lstlisting}[caption=Java code example showing how to send and receive commands to a NFC smart card., label=lst:NFCLibraryExample,escapechar=å]

public class PayloadActivity extends AppCompatActivity
    implements NFCSmartcardControllerInterface {
    CommunicationController cc = new CommunicationController();
    ...

    private void initNFCCommunication(){


        String AID = "0102030405060708090007";
        String hexMessage = "95404F3FB1";
        String INS = "06";
        String p1 = "00";
        String p2 = "00";
        cc.setupNFCController(this, this);
        cc.initNFCCommunication(AID, INS, p1, p2, hexMessage);
    }

    å@Overrideå
    public void nfcCallback(final String completionStatus){
        if(!completionStatus.equals("OK")){
            return;
        }
        StorageHandler stHandler =
            new StorageHandler(getApplicationContext());
        String response =
            stHandler.readFromFileAppDir(
                FilePaths.tempStorageFileName
            );
    }
}

\end{lstlisting}

In order for the library to perform an asynchronous transaction the library will temporary save the responses from the cards to a file only accessible by the running application. To retrieve the data the current activity should use the included \texttt{StorageHandler} class as used in listing \ref{lst:NFCLibraryExample}. The library also provides the class, \texttt{Converter}, for converting between Strings, hex and byte arrays.

\paragraph{Pre-implemented methods}\mbox{}\\
Recall the areas we want to cover from the beginning of the chapter. The functionality we have implemented so far are:

\begin{itemize}
    \item Bind smart card to mobile device.
    \item Encrypt/decrypt data using RSA key on card.
    \item Encrypt/decrypt data using AES key on card.
    \item Get public key of the smart card.
    \item Sign data using the public key of the smart card.
\end{itemize}

To use these functionalities one would only need to create an Android \texttt{Activity}, invoke either \texttt{\allowbreak setupNFCController(...)} or \texttt{setupmSDController(...)} (depending on smart card), and utilize the desired methods. In figure \ref{fig:instantiateflow} we can see how \texttt{CommunicationController} is designed to be the abstraction layer between Android activities and smart cards.


\begin{figure}[h!]
  \caption{Abstraction layer between Android activities and smart cards.}
  \label{fig:instantiateflow}
  \centering
    \includegraphics[width=0.95\textwidth]{images/Instantiate_flow.png}
\end{figure}

The methods available are:

\begin{itemize}
    \item public void disableNFC(...)
    \item public void signData(...)
    \item public void cryptoRSA(...)
    \item public void cryptoAES(...)
    \item public void getCardPubMod(...)
    \item public void getCardPubExp(...)
    \item public void bindingStepOne(...)
    \item public void bindingStepTwo(...)
    \item public void bindingStepThree(...)
\end{itemize}

The methods and their functionality should be self-explanatory except for the binding process. The binding process is designed in three steps. First step is to ask the smart card if it requires a PIN-code and how many attempts are left. Second step requires a PIN-code and if this is correct the smart card application will move on to step three. The last step is sending the public key of the mobile device and getting the verification package from section \ref{sec:proposedSolution}. More discussion on this matter in section \ref{sec:bindingcardandphone}.

Listing \ref{lst:NFCSigning} shows how an activity can use the \texttt{CommunicationController} to sign a simple message. The \texttt{signData(...)} method takes 3 parameters: CommunicationType, AID and the hex message to be signed. In the method \texttt{nfcCallback(...)} the developer are free to do whatever they want. Typically it is a good idea to check what the completeionStatus string is before trying to fetch the response data. Read appendix \ref{app:b} for all method signatures.

\begin{lstlisting}[caption=Java code example showing how to send sign a message using a NFC smart card., label=lst:NFCSigning,escapechar=å]

public class SigningActivity extends AppCompatActivity
    implements NFCSmartcardControllerInterface {
    CommunicationController cc = new CommunicationController();
    ...

    private void initNFCCommunication(){


        String AID = "0102030405060708090007";
        String message = "This message must be signed.";
        String hexMessage = Converter.StringToHex(message);
        cc.setupNFCController(this, this);
        cc.signData(CommunicationType.NFC, AID, hexMessage)
    }

    å@Overrideå
    public void nfcCallback(final String completionStatus){
        if(!completionStatus.equals("OK")){
            return;
        }
        StorageHandler stHandler = new StorageHandler(
          getApplicationContext()
          );
        String response = stHandler.readFromFileAppDir(
          FilePaths.tempStorageFileName
          );
    }
}

\end{lstlisting}

\paragraph{Available classes}\mbox{}\\
Figure \ref{fig:classdiagram_simple} provides a simplified and technical overview of how the Android side of the library is built up. The classes seen in the simplified class diagram \ref{fig:classdiagram_simple} shows the three controller classes, \texttt{CommunicationController}, \texttt{NFCSmartCardController} and \texttt{MSDSmartCardController}, we have implemented along with their method signatures. The enum class \texttt{CommunicationType} is used for indicating which type of communication one wishes to invoke when using the methods of \texttt{CommunicationController}.

The two controllers, \texttt{NFCSmartCardController} and \texttt{MSDSmartCardController}, can be used directly. If this is done, one cannot use the pre-implemented methods in the Android library. The corresponding methods on the smart card are still available, but with this approach one will need to construct the APDUs manually and conform to their message structure. This approach is still a viable way of using the framework, as both controllers provide a method for sending data to the smart card using the parameters: card AID, INS, P1, P2 and Payload. The method ensures that the APDU being sent to the smart cards conform to the ISO standards of APDUs, but the developer run the risk of invoking functionality that does not exist or with wrong parameters.

Depending on which type of communication type one wishes to use, one will need to implement the corresponding interface in the Android Activity as we discussed earlier. This can be seen in action in listing \ref{lst:NFCSigning}.

The complete class diagram for the Android library can be found in Appendix \ref{app:c}, figure \ref{fig:classdiagram_extended}. This figure contains all implemented classes, interfaces and enums. If there is a need to implement custom controllers for smart card communication it is possible to use the classes which communicates directly with the Gemalto library or the Android.nfc package. A potential future need might be to add communication with a new type of smart card. Classes such \texttt{Converter}, \texttt{StorageHandler}, \texttt{ApduStatics} and \texttt{FilePaths} provide necessary functionality for APDU construction, message parsing and asynchronous message handling. The package diagram in figure \ref{fig:package} shows how the specific smart card controllers include these classes. %The package diagram, figure \ref{fig:package}, shows how  the packages in the library are structured.

\begin{figure}[h!]
  \caption{Simplified class diagram for Android Library.}
  \label{fig:classdiagram_simple}
  \centering
    \includegraphics[width=1.25\textwidth, angle =90]{images/Class_Diagram.png}
\end{figure}

We have deliberately not made any of the classes protected, meaning that all classes can be instantiate anywhere. This allows new frameworks or new functionality to build on our work without having to edit the library. Of course, this opens up the possibility for uses which are not possible to execute. For example sending APDUs to a smart card without opening the channel. We expect developers that chooses this path to have an understanding of how smart card communication must be built up.
