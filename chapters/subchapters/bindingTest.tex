% !TEX root = ../../Article.tex
\section{Binding card and mobile device}
\subsection{Motivation}
In section \ref{sec:bindingcardandphone} we describe and motivate a solution where the we want to bind a mobile device to a smart card. We find it important to do an empirical test of the solution due to the fact that we have time constraints regarding running time. The parts of the binding process we want to test is:

\begin{itemize}
    \item Mobile device is able to transfer public key to smart card.
    \item Smart card can store the public key from the mobile device.
    \item Smart card can generate the \textit{verifaction package} from section \ref{sec:proposedSolution}.
\end{itemize}

\subsection{Implementation}
In light of the parts of the binding process we want to test the outline of the implementation will be:

\begin{enumerate}
    \item Mobile devices asks smart card if its authenticated with PIN.
    \item Smart card responds with yes/no and amount of PIN tries remaining.
    \item Mobile device prompts user for PIN and sends it to the smart card.
    \item Smart card verifies PIN or the process skips back to step 2.
    \item Mobile device sends it public key to the smart card.
    \item Smart card generates the verification package (figure \ref{fig:h0}).
    \item Smart card sends the verification package to the mobile device.
\end{enumerate}

Due to the nature of smart cards we will need to hard code this protocol into the android application and the smart card application. We will use the P1 byte in the APDU to define which step of the process we are in, respectively:
\begin{itemize}
    \item \texttt{0x01} for step 1 and 2.
    \item \texttt{0x02} for step 3 and 4.
    \item \texttt{0x03} for step 5,6 and 7.
\end{itemize}

To manage PIN verification and management we will use \texttt{OwnerPIN} class from the JavaCard framework \cite{ownerpin}. After the PIN is set on the card during installation we can utilize the methods: \texttt{isValidated()} for checking if the right PIN is already provided, \texttt{check(byte[] pin, short offset, byte length)} for checking if the provided PIN is correct and \texttt{getTriesRemaining()} for getting remaining tries.

All public keys are represented as byte arrays when they are transmitted between the mobile device and the smart card. The format for the byte array is \textbar{}ModulusLength\textbar{}Modulus\textbar{}ExponentLength\textbar{}Exponent\textbar{} to allow for extension of key length and dynamic importing. After the APDU is received the public key is stored in a \texttt{RSAPublicKey} object for storage.

When we construct the verification package on the smart card we have to convert the public key of the smart card over to the same format as the mobile device's public key. After we have transformed all relevant keys (see figure \ref{fig:h0}) to byte arrays we can finally put them together, sign the package and encrypt it with the public key of the server.

\subsection{Tests and results}
\paragraph{Configuration 1}\mbox{}\\
This configuration uses 512-bit keys as the public key between the mobile device and smart card. Although we wish to use keys that are greater than 2048-bit we feel that this is a good starting point.
\begin{itemize}
    \item NFC card
    \item 512-bit mobile device public key
    \item 512-bit smart card public key
    \item 2048-bit server public key
    \item 128-bit AES key
\end{itemize}

\paragraph{Configuration 2}\mbox{}\\
This configuration uses 2048-bit keys as the public key between the mobile device and smart card to simulate a real-world example.
\begin{itemize}
    \item NFC card
    \item 2048-bit mobile device public key
    \item 2048-bit smart card public key
    \item 2048-bit server public key
    \item 128-bit AES key
\end{itemize}

\paragraph{Installation test}\mbox{}\\
We want to find out if initializing the keys we need for generating the verification package affects the installation of the smart cards. We will deploy and install the smart card application using GlobalPlatfromPro and time how long it takes. Our test set is 100 tests.

\begin{table}[h!]
\caption{Time required to install the application on the smart card.}
\label{tbl:bindingInstall}
\centering
    \begin{tabular}{ | c | c | c | c |}
        \hline
        \thead{Configuration}
        & \thead{Average}
        & \thead{Maximum}
        & \thead{Minimum} \\ \hline

        1 & 24,28s & 59s & 9s\\ \hline
        2 & 39,42 & 95s & 13s\\ \hline
    \end{tabular}

\end{table}

\paragraph{Run test}\mbox{}\\
In this test we will generate the verification package. We will skip the parts of the process involving user input (PIN code).
\begin{table}[h!]
\caption{Time required to generate the verification package on the smart card application.}
\label{tbl:bindingRun}
\centering
    \begin{tabular}{ | c | c | c | c |}
        \hline
        \thead{Configuration}
        & \thead{Elapsed time} \\ \hline

        1 & 1,22s\\ \hline
        2 & N/A\\ \hline
    \end{tabular}

\end{table}

\subsection{Limitations}
After implementing the solution some limitations and problems became apparent. It is important to consider these when determining the effectiveness of the solution.

\paragraph{Signing not working with 2048-bit key size}\mbox{}\\
We encountered a bug when switching to ``configuration 2''. The smart card application started actually crashing, as not in responding with error codes, but actually crashing and losing power. We managed to narrow it down to the exact line of code the application crashes on (see listing \ref{lst:signing2048}).

At first we suspected that the packet was too big or some mismatch in the parameters. The function still crashed with a smaller packet and all parameters are of the correct length/type/value. It is also worth noting that the 2048-bit key are properly initialized and working. Identifying the problem is hard considering the only clues we have are:

\begin{itemize}
    \item Signing crashes on different input data with 2048-bit key.
    \item No error codes - Hard crash.
\end{itemize}

After searching the Internet for others with the same problem it became apparent that others have had troubles with signing. Some report that the running time of their smart cards increase drastically when using to 2048-bit keys \cite{signingtrouble2}. We are not able to find any official sources on these issues, but there seems to be an underlying problem or bug in JavaCard.

\begin{lstlisting}[caption=Java Card failed signing., label=lst:signing2048, escapechar=!]
public class cardApplication extends Applet implements !\allowbreak!ExtendedLength{

    private RSAPublicKey k;
    private Signature sig;

    ...

    private cardApplication() {
        keys = new KeyPair(
            KeyPair.ALG_RSA,
            KeyBuilder.LENGTH_RSA_2048);
        k = (RSAPublicKey) keys.getPublic();
    	sig = Signature.getInstance(
            Signature.ALG_RSA_SHA_PKCS1,
             false);

        ...
    }

    public void process(APDU apdu) {
    	...

        !\colorbox{highlight}{signatureSize = sig.sign(}!
            !\colorbox{highlight}{packet,}!
            !\colorbox{highlight}{(short) 0,}!
            !\colorbox{highlight}{(short) packetSize,}!
            !\colorbox{highlight}{h0Unencrypted,}!
            !\colorbox{highlight}{(short) 0);}!

        ...
    }

    ...
}
\end{lstlisting}

\paragraph{Out of memory}\mbox{}\\
We are dealing with many different byte arrays in our solution, and due to the design of smart cards we will need to allocate memory for these byte arrays when the smart card is initialized. The size of these byte arrays are dependent on the key sizes we use. Allocating 5 byte arrays of length 500 (which is plenty for 2048-bit keys) is independently fine, but if the smart card application allocates a lot of resource alongside this solution we need to be careful of running out of memory.

It is important to include hidden memory sinks such as the \texttt{RSAPublicKeys} and the encryption done by the \texttt{Cipher} class. We can just as easily run out of memory by changing our key lengths as when allocating byte arrays.

\paragraph{Code rigidity}\mbox{}\\
Because of the design of smart cards our implementation is heavily hardcoded, and as a results is very rigid to change. For instance if the specification of the verification package change it will require excessive work to reflect the changes. One will have to decide make a decision whether or not this will happen often enough to counter the potential benefits of using smart cards.

\subsection{Conclusion}
We have shown that the smart card area of responsibility in the binding process is possible to perform. Test results show that the installation process can be rather long (ranging from 9 seconds to 95 seconds), but that the verification package generation is effective ({\raise.17ex\hbox{$\scriptstyle\sim$}}1 second).

The installation process is a one time process and the keys we generate during it are needed for other cryptographic operations on the card. We find it safe to assume that this process is a cost we can afford. The verification package generation process has such a low processing time that we can also assume that we can afford it. Our conclusion is that the binding process is feasible and that the reward, that smart card and mobile device is locked to each other, greatly outscales the costs involved.
