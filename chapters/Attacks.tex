% !TEX root = ../Article.tex
\chapter{Mobile threats}
Mobile threats and attack vectors are numerous and before looking into if smart cards can help mitigate or alleviate we will need to identify and characterize them.
\section{Infected device}
The most obvious threat to mobile devices is when the mobile device itself is infected with virus or malware. The type of virus or malware can vary, some are harmless and serve more as an annoyance or trying to trick the user into visiting bogus websites, but some are more malicious and will access private files and information. From a security stand-point it is a disaster if a virus or malware is able to read and modify data which is otherwise confidential.

Often the user will not know that their mobile device is infected and some viruses or malware are very hard to detect by anti-virus. The ``2015 Cheetah Mobile Security Report'' \cite{cheetahSec} reports that the number of viruses on Android devices exceeds over 9,5 million and that the problem is growing. Taking into concideration that the mobile device we are operating on may be infected is of the utmost importance when designing and developing applications.

\section{Lost or stolen device}
An attacker may gain physical access to the users mobile device through theft or simply that the user misplaced the mobile device. With physical access to the device an attacker would be able to retrive data from the device. A common defence against this is encrypting the data on the device, but this requires the keys to be stored somewhere securely. If the keys are not stored securely the result is that the data on the device will fall into the wrong hands.

\section{Unsecure communication channel}
\label{sec:unsecureCommunication}
Communication is a vital part of modern systems; data is sent between devices and between devices and servers. Sensitive data requires a secure communication channel which cannot be tapped into by a third party. Secure communication on public networks involves agreeing upon encryption keys which the data should be encrypted with before being sent. Encrypting the communication channel will protect against man-in-the-middle attacks, but this requires both parties to authenticate themselves as encrypting the data won't help if you are sending the data directly to the attacker. More on this in section \ref{sec:authenticationChallenges}.

\section{Authentication challenges}
\label{sec:authenticationChallenges}
%Are people who they say the are?
As mentioned in section \ref{sec:unsecureCommunication} a secure communication channel is useless if you are sending the data directly to the attacker. A vital part of security is being able to authenticate the parties in a transaction. If the attacker is able to impersonate another party by installing fake certificates on the mobile device or by tricking the user into communicating with the attacker the consequences can be of significance.

\section{Unsecure applications}
An often overlooked attack vector is badly implemented applications on the mobile device. This may include memory leaks, weak cryptography, open for code injections and plainly exposing private data to third parties. In rare cases an application with flaws may expose other applications for attacks, but there exists countermeasures to this, for instance that all applications run in their own sandbox.
