% !TEX root = ../Article.tex
\chapter{Conclusion}
\label{ch:conclusion}

\section{Research questions}
\begin{itemize}
  \item \textit{``What are the limitiations of smart cards in the context of hardware?''}\\
  There hardware limitations of smart cards are very dependent on which smart card you are using. Generally smart cards are limited when it comes to computing power and this has a huge effect on how resource intensive operations you are able to perform on the smart card. Secure cryptography are very resource intensive and our test results show that encrypting large amounts of data is unfeasible, both for public-key cryptography (RSA) and symmetric-key cryptography (AES). We performed tests regarding transfer speed and they show that the throughput of input/output are limited and that we often run the risk of running out of memory with large amounts of data.
  \item \textit{``What are the limitiations of smart cards applications?''}\\
  We opted for using JavaCard as our programming language. Our version of JavaCard does not support advanced datatypes. This combined with the fact that all data being sent to/from the smart card is byte values creates a rigid environment with hardcoded values. JavaCard does not support standard garbage collection and thus applications needs to be extra careful when allocating memory.
  \item \textit{``What are the types of use cases we are able to solve and strengthen the security of using smart cards?''}\\
  Even though smart cards have some areas with limitations we are able to identify use cases where smart cards can be applicable. In use cases involving cryptography a smart card can store the keys securely as well as encrypt/decrypt small amounts of data. Due to the fact that smart cards are tamper proof, meaning that you are not able to extract data (keys), we are confident that smart cards can alleviate threats such as stolen mobile devices and insecure communication channels.\mbox{}\\\\  We believe that smart cards can add an extra layer of security for areas that are already solved. In this thesis we described and analyzed a solution where we used smart cards as a basis for policy enforcement. Our comprehension is that this type of solution in conjuction with traditional policy enforcement systems will take security to a new level.
\end{itemize}

\section{Experience}
After working with smart cards for roughly one year we have made quite a few experiences that are worth sharing.

\paragraph{Getting started with smart card programming}\mbox{}\\
Getting started with smart card programming can be difficult. After a few years with object-oriented programming most programmers will start to get comfortable using ``quality of life'' classes such as \texttt{ArrayList} and \texttt{Enum}. That world gets turned up-side down when moving to Java Card. One is  thrown back to an older Java version and as a programmer you will need to rethink how you solve problems and structure solutions. Most notable is the fact that all incoming data is in the form of a \texttt{byte} array that must be mapped to the correct datatypes. Missing functionality such as standard Java garbage collection and standard data types (\texttt{int}, \texttt{double}, etc.), makes Java Card programming cumbersome and requires time to adapt to.

\paragraph{Debugging smart card applications}\mbox{}\\
Debugging smart card applications differs from standard debugging. Normally when debugging you are able to insert breakpoint, inspect variables and monitor resource usage. The nature of smart cards is to be a secure and closed environment and thus it is hard to monitor how an application behaves. The debugging method we have available is: deploy the application, send data to it and see what the response is. If the response does not match expected output the best way to debug is creating manual breakpoints, i.e., add a line of code returning the value of variables and try to figure out where the error might be. Sometimes the smart card encounter runtime exceptions and sends a 2 byte response that is mapped to an error message \cite{javacardErrors}.

This type of debugging environment is exhausting and is requires a lot of resources. Often we spent time trying to pinpoint an error only to later found out that the error code we got had nothing to with the actual problem. This was especially notable with errors regarding memory usage. One of the best advices concerning smart card debugging is: ``Test often with a big array of test data.''.

\paragraph{JavaCard documentation}\mbox{}\\
The documentation available is very technical. This is not by any means a bad thing, but it does require developers to understand smart cards fully before using the documentation. When comparing Android and JavaCard documentation, it is very apparent that Google has put a lot of effort into having an educational approach to the concepts before diving into the technical aspects. In the JavaCard documentation there are very few examples of usage, and we spent a lot of time trying to figure out how to properly use classes and methods.

The gap between software and hardware is very apparent in the JavaCard documentation. We often encountered functionality that was supposed to work, but did not work on our smart cards. The result of this was that when we encountered bugs, we did not know if it was a programming mistake or simply not supported by our smart cards. The best example of this was when we tried to use the \texttt{Cipher} class with algorithms that proved to not work on our smart cards (section \ref{sec:symmetricTest}).

\paragraph{Deploying smart card applications}\mbox{}\\
Deploying smart card applications to a smart card is a time consuming task. This became very apparent when working with micro SD smart cards. Deploying a new version to micro SD required us to: remove mSD from mobile device, insert mSD into computer, run install script, wait on install script to finish, insert mSD into mobile device. Following this procedure once in a while is not a big inconvenience, but in context of debugging it become very tedious to spend 1,5 minutes switching around the mSD card and waiting for the install script.

\section{Future work}
\label{sec:future}
The research we have presented in this thesis are a good starting point for developing custom security applications on the Android platform in conjunction with smart cards. The test cases we have looked into point to that micro SD cards have better performance than NFC cards, but our micro SD cards did not support extended APDUs and thus we cannot confirm that micro SD are better than NFC cards. More work on micro SD cards must be performed in order to confirm these suspicions.

We encountered numerous bugs and limitations when working with smart cards which we did not initially predict, and as a result the Android library and the smart card application is not as polished and refined as we had hoped it would be. This includes adding more pre-implemented functionality, refactoring code to be more readable and optimize code to achieve better performance. Additionally we believe it would be beneficial to look at the possibility of not being dependent on the Gemalto framework for micro SD card communication \cite[\textit{SEEK for Android}]{SEEK}.

Our evaluations of the proposed solutions are based on protocol analysis and proof of concept. It would be beneficial to perform penetration tests on the outlined solutions to confirm that: \textit{a)} We are able to implement all parts of the solution. \textit{b)} We can show that the solution is methodically tested against known attacks in today's society.
